\documentclass[12pt, oneside, a4paper]{article}
\usepackage[utf8]{inputenc}
\usepackage[margin=1in]{geometry}
\usepackage{amsfonts, amsmath, amssymb}
\usepackage[none]{hyphenat}
\usepackage{fancyhdr}
\usepackage{graphicx}
\usepackage{float}
\usepackage[nottoc, notlot, notlof]{tocbibind}
\usepackage{xcolor}
\graphicspath{ {figures/} }
\pagenumbering{arabic}



\pagestyle{fancy}
\fancyhead{}
\fancyfoot{}
\fancyhead[C]{}
\fancyfoot[C]{\thepage}


\begin{document}
%\begin{titlepage}
\begin{center}
    \vspace{0.1cm}
    \begin{Large}\textbf{Securing Online Transactions}\end{Large}\\
    \vspace{0.8cm}\begin{large}
    \textit{A project submitted in partial fulfillment of the\\
    Requirements for the award of the degree of}\\
    \vspace{0.8cm}
    \textbf{Bachelor of Technology\\ in}\\
    \textbf{\colorbox{yellow}{COMPUTER SCIENCE AND ENGINEERING}}\\
    \vspace{1cm}
    \includegraphics[width=0.3\textwidth]{Clg_logo.png}\\
    \vspace{0.8cm}
    \line(1,0){400}\\\vspace{0.3cm}
    Submitted by:\\
    \textbf{Mr. Karanbir Singh Pannu}\\Rol No.: 12011011\\
    \vspace{0.5cm}
    \line(1,0){150}\\
    \vspace{0.5cm}
    Supervised by:\\
    \textbf{Dr. Mukesh Mann}\\
    Assistant Professor\\
    \line(1,0){400}\\
    \vspace{2cm}
    \textbf{INDIAN INSTITUTE OF INFORMATION TECHNOLOGY,\\
    SONEPAT -131021, HARYANA, INDIA}\\\end{large}
\end{center}
\thispagestyle{empty}
%\end{titlepage}
\pagebreak
%2nd Page
\begin{center}
    \begin{Large}\textbf{ACKNOWLEDGEMENT}\end{Large}
\end{center}\\
\vspace{1cm}
I would like to extend my sincere thanks to Dr Mukesh Mann for providing us with the glorious opportunity to work on the magnificent project Securing Online Transactions and for their precious assistance at every juncture of our journey in bringing this project to reality. I find myself lucky to be the recipient of his advice and encouragement.

I cannot ever I think, succeed in verbalising the support and inspiration bestowed upon me by my parents who despite their busy schedules never passed over any moment where they could succour me. I am very grateful for their highly cherished support during this project.

I consider myself lucky to have secured a chance to make this project as it imparted a great deal of knowledge about real life problems and the methods to approach and finally attempt to solve them. I learned a lot about many tools which were initially new to me to handle such projects. 

Last but not the least, I am very thankful to my team members for providing me with the honor of working with them as a group. Their sincerity and inclination towards the project was a major source of inspiration for me. They spared no effort to help every time I stumbled upon an obstacle in this journey. It was a real pleasure working with them.
\vspace{2cm}\\
\textbf{Karanbir Singh Pannu(12011011)}
\setcounter{page}{2}
\pagebreak

%3rd page
\begin{center}
    \begin{Large}\textbf{SELF DECLARATION}\end{Large}
\end{center}\\
\vspace{1cm}
I hereby declare that work contained in the project titled "\textbf{SECURING ONLINE TRANSACTIONS}" is original. I have followed the standards of research/project ethics to the best of my abilities. I have acknowledged all sources of information which I have used in the project.
\vspace{4cm}\\
Name: Karanbir Singh Pannu\\
Roll No.: 12011011\\
Department of Computer Science and Engineering,\\
Indian Institute of Information Technology,\\
Sonepat-131201, Haryana, India.
\pagebreak

%4th page
\begin{center}
    \begin{Large}\textbf{CERTIFICATE}\end{Large}
\end{center}\\
\vspace{1cm}
This is to certify that Mr. Karanbir Singh Pannu has worked on the project entitled "SECURING ONLINE TRANSACTIONS" under my supervision and guidance.
\vspace{0.5cm}\\
The contents of the project, being submitted to the Department of Computer Science and Engineering, IIIT, Sonepat, for the award of the degree of B.Tech in Computer Science and Engineering, are original and have been carried out by the candidate himself. This project has not been submitted in full or part for the award of any other degree or diploma to this or any other university.
\vspace{4cm}\\
Dr. Mukesh Mann\\
Supervisor
\vspace{5cm}\\
Department of Computer Science and Engineering,\\
Indian Institute of Infomation Technology,\\
Sonepat-131201, Haryana, India
\pagebreak

%9th page
\begin{center}
    \begin{Large}\textbf{Abstract}\end{Large}
\end{center}\\
\vspace{1cm}
Name of the student: \textbf{Karanbir Singh Pannu}\vspace{0.1cm}\\
Roll No.: \textbf{12011011}
\vspace{0.3cm}\\
Degree for which submitted: \textbf{B.Tech(CSE)}\vspace{0.2cm}\\
Department of \textbf{Computer Science and Engineering, IIIT Sonepat.}\vspace{0.2cm}\\
Project Title: \textbf{SECURING ONLINE TRANSACTIONS.}\vspace{0.2cm}\\
Name of thesis supervisor: \textbf{Dr. Mukesh Mann}\vspace{0.2cm}\\
Month and year of thesis submission: \textbf{December 2021}
\vspace{2cm}\\
E-banking is receiving an ever increasing popularity due to better service and a competitive advantage to the bank. However, this popularity has attracted fraudulent behavior of people to engage in stealing online. Probably, this type of banking has not fully established the necessary security requirements to ensure safe transactions due to which frauds find out doors to sneak in and commit crimes. There could be a higher motivation factor in online fraud as fraudsters only see some figures go from one place to another and they are distant from the fact of robbing someone of physical currency.

\vspace{0.5cm}
Photo verification before transaction presents an added layer of security in online banking. Whenever a user's bank account is accessed to make a payment in favour of someone, the website clicks a picture of the transactor and sends and email to the account's owner who can then decide whether to permit or deny the transaction with a click. This method would discourage people with deceitful intentions to meddle with anyone's online assets. This system would also make it impossible for bots or viruses on the internet to infiltrate someone's bank account, therefore establishing a secure and trustworthy system of transactions.

\vspace{0.5cm}
This is a simple yet reliable defense against online frauds as it requires the conscious presence of the owner to validate a transaction. This system can be deployed with ease on basically all devices that can carry out online banking and are equipped with a camera for photo capture. Although, not all devices have an inbuilt camera, this system would be welcomed by numerous users who have built in cameras in all their devices handling their online transactions as it would incapacitate a range of devices from accessing their account.
\pagebreak

\begin{center}
    \begin{Large}\textbf{List of Abbreviations}\end{Large}
\end{center}\\
\vspace{1cm}
\begin{table}[H]
    \begin{tabular}{c c l}
    AI & - & Artificial Intelligence\\
    CGR & - & Compound Growth Rate\\
    P2P & - & Peer-to-Peer\\
    Q & - & Quarter\\
    ATM & - & Automated Teller Machine\\
    ML & - & Machine Learning\\
    IP & - & Internet Protocol\\
    HMM & - & Hidden Markov Model\\
    OTP & - & One Time Password\\
    WAPOL & - & West Australian Police\\
    LSTM & - & Long Term-Short Memory\\
    \end{tabular}
\end{table}
\pagebreak

\listoffigures
\pagebreak
\begin{large}
\begin{itemize}
    \item Acknowledgement & . . . . . . . . . . . . . . . . . . . . . . . . . . . . . & 2
         \item Self Declaration & . . . . . . . . . . . . . . . . . . . . . . . . . . . . . .  & 3
         \item Certificate & . . . . . . . . . . . . . . . . . . . . . . . . . . . . . . . . . & 4
         \item Abstract & . . . . . . . . . . . . . . . . . . . . . . . . . . . . . . . . . . & 5
         \item List of Abbreviations & . . . . . . . . . . . . . . . . . . . . . . . . . . . & 6
         \item List of Figures & . . . . . . . . . . . . . . . . . . . . . . . . . . . . . . . & 7
         \item Table of Contents & . . . . . . . . . . . . . . . . . . . . . . . . . . . . . & 9
\end{itemize}

\end{large}
\pagebreak
\tableofcontents
\pagebreak

%10th page
\begin{Large}
\begin{center}
\textbf{CHAPTER\\\section{INTRODUCTION}}\\
\line(1,0){300}\\
\end{center}
\end{Large}
\vspace{0.5cm}
\subsection{Introduction}
Online banking is an electronic payment system, acceding the user to direct financial transactions via the internet, also known as internet banking or web banking. It was made to be convenient and time-saving, providing users with an exemplary banking experience and real time problem resolution services. It removes the necessity of a user to visit the bank physically to make payments, and allows its users to supervise their funds from anywhere through the internet.

Digital fraud is not a mysterious term to the world as businesses have been facing digital frauds since the advert of e-commerce in the 1990s and its threat only ever increases with time. With the increasing use of online payments and transactions due to the current global situation, chiselers take on various tactics to purloin someone's funds. These thefts could be carried out by stealing someone's credentials or directing someone to a payment gateway, rigged in their favor. Many time, these thieves pilfer meagre amounts of currency from multiple users making it hard for the user's detection or worth her/his time.

The aim of this project is to forestall such banking frauds by the manoeuvre of verification of the person effectuating the transaction, by the owner. The primary objective of the project is to implement photo verification of the transactor via an email sent to the account holder through a web application. When someone logs into a bank account using the particular credentials, which could have been obtained by fair or unfair means, the application supervises that if the transaction amount exceeds a specific sum, the picture of the transactor would be taken and mailed to the account holder who would then decide whether it is an authorized payment or there is some foul play at work.

Identity is the attribute of identical, the correspondence of one thing with another. Identity verification ensures that there is a real person behind a process and proves that the one is who he or she claims to be, preventing both a person from carrying out a process on our behalf without authorization, and creating false identities or commit fraud. This project takes the traditional face-to-face process and applies it to online platform eliminating the need of physical correspondence. The person launches the website, allows the use of camera and is required to prove his authorization to carry out the specific task.

This project is highly scalable as in case of heavy traffic or for management of organizations with a colossal number of clients, the web app could be equipped with an AI program to validate its processes. It is cost effective and user friendly and can be deployed on almost any conventional device capable of carrying out such transactions. It eliminates the threat of bots or viruses from committing thefts of such kind. Overall it provides a reliable medium for online transactions and other high security data manipulation processes alike.

\pagebreak
\subsection{Problem Outline}
\vspace{0.5cm}
\subsubsection{Online Frauds}
The global online banking market size was valued at \$11.4 billion in 2019 and is projected to reach \$31.81 billion by 2027, growing at a CGR of 13.6\% from 2020 to 2027. In India, nearly 6900 cases of online banking frauds were registered in the year 2017-2018. As per data maintained by the National Crime Records Bureau, 3,466 and 3,353 cases of online frauds were registered in 2017 and 2018, respectively. Despite the entire country being under lockdown, the number of cybercrime complaints have tripled during this period due to the increased use of online services. Data released by Delhi Police showed that 3,430 complaints were received in May compared with 1,260 in January 2020. While 60\% of the complaints were regarding financial fraud, 20\% were of online harassment.

Figure 1 shows online bank frauds in different sectors in the last ten years.

\begin{figure}[H]
\begin{center}
\includegraphics[width=0.7\textwidth]{Figure1.png}\\
\caption{Online Frauds by amount}{Source: The Hindu.}
\label{fig:Figure 1}
\end{center}
\end{figure}

Even though consumers and businesses are getting more comfortable with navigating the digital world, fraudsters are becoming just as adept at finding ways to take advantage of these online trends. According to the Javelin 2021 Identity Fraud Study: Shifting Angles, there was \$56 billion in combined losses from identity fraud & identity fraud scams in 202, accompanied by a significant increase in identity theft scams.

Another fraud management trend pertains to digital wallets and peer-to-peer (P2P) payments. Identity fraud scams claimed 17.5 million digital wallet and P2P fraud victims. Victims of stimulus-refund-related fraud, unemployment money mule scams, and identity fraud scams all owned digital wallets like Apple Pay and Samsung Pay, and P2P products like PayPal, Square and Zelle. 

Feedzai's Quarterly Financial Crime Report, analysing over 12 billion transactions, identifies trends in spending and in fraud attempts to show that this past quarter, as consumer activities increased, fraudsters attempted to hide their fraudulent transactions in legitimate banking. In fact, combining all banking fraud  – internet, telephone, and branch – attacks grew a whopping 159\% in Q1 2021 compared to Q4 2020. 

Online banking made up 96\% of all banking transactions and it accounted for 93\% of all fraud attempts in Q1 2021. This leaves in-branch and telephone banking to make up the remaining 4\%. And while the numbers are smaller, in-branch banking did increase by 442\% this quarter compared with the last as a result of eased lockdown restrictions as businesses begin to open for trade. In addition, telephone scammers upped their efforts and the report shows a 728\% increase in telephone banking fraud.

\subsubsection{Methods of Online Frauds}
Following are some of the most common ways of online frauds in the recent time. Most of the victims are poor users of E-banking and easily believe someone posing as a bank official.
\begin{itemize}
    \item Phishing.
    \item Spam.
    \item Spyware.
    \item Card Skimming.
    \item ATM Skimming.
    \item Hacking.
    \item Identity theft.
\end{itemize}

\subsubsection{Challenges in Fraud Detection}
To detect a potential banking fraud is not an easy task. Some of the challenges faced during fraud detection are:
\begin{itemize}
    \item \textbf{Changing fraud patterns over time - }This is one of the sturdiest to address as fraudsters are always on a lookout for advanced ways to trick the system. This requires the fraud detection programs to be updated with the evolved patterns which decreases its efficiency.
    \item \textbf{Class Imbalance - }More or less, only a small fraction of customers have fraudulent intentions which introduces an imbalance in the classification of fraud detection models. The side effect being a penurious experience for genuine customers, since catching the fraudsters usually involves declining some legitimate transactions.
    \item \textbf{Model Interpretations - }This limitation is associated with the concept of comprehension since models typically give a score indicating whether a transaction is likely to be fraudulent or not — without explaining why.
\end{itemize}
\pagebreak
\subsection{Project Objective}
Cybercrime is one of the major problems faced by the countries across the globe these days. It includes unauthorized access of information and invasion of privacy to obtain passwords, bank data etc. of any person with the use of internet. With the increasing demand of online payment in almost every sector of business, online frauds need to be addressed sternly. 

The vision of this project is to make online banking a safe and secure payment medium through the employment of identity validation. Identity validation is an important requirement in most processes and procedures, whether offline or online. Identity validation lays down trust in the consumer that his funds are being handled fairly. It proves that there is a real person behind the process and is indeed the one he or she claims to be. It is a simple procedure which could keep one's account safe and at the same time, not derange genuine customers as it only takes a couple of moments to complete the whole validation process. Although simple, it is very sound as the command rests solely in the hands of the account holder.

According to Harry (2002), hackers and crackers directly attack servers to commit cybercrimes such as stealing passwords, credit card information and other confidential or secret information; to intercept transactions and communications and commit online frauds\cite{Project_Objective}. Therefore, theft of credentials of users is the major problem to be looked into to solve online frauds. A simple technique of identity validation can save someone from online theft in case his or her credentials get stolen.

\vspace{1cm}
\textbf{Major Objectives}
\begin{itemize}
    \item To create a safe payment gateway for online transactions.
    \item To implement photo verification as a type of identity validation.
    \item To fend off fraudsters from misusing customer's credentials to commit online crimes.
    \item To allow the user to change his/her password in case of suspicious activity on the account.
    \item To place the complete control of a user's online funds into his/her own hands.
\end{itemize}

\vspace{1cm}
Although the main goal of the project is photo verification to validate a transaction, the user is also allowed to change his login credentials in case of detection of hacking or potential theft.

\pagebreak

\subsection{Methodology}
The steps given below provide the solution to the problem of online frauds using identity validation. The solution involves verification of the transactor by the end user in real time and the ability to change the password in case of suspected foul play.
\subsubsection{Photo Verification(Working)}
\begin{enumerate}
    \itemsep0em
    \item User logs into the account who can be a genuine user or a fraud.
    \item The transaction details are entered.
    \item The web app takes a picture of the transactor and sends it on the account holder's email.
    \item The end user can then validate the transaction in which case the transaction is carried out normally.
    \item The end user can deny the transaction which cancels the payment process and the user can also change his account password in case of a suspected fraud.
\end{enumerate}

The above process can be completed in nearly the same time as a normal transaction to conserve the genuine user's time.
\vspace{1.5cm}

\begin{figure}[H]
\begin{center}
\includegraphics[width=0.8\textwidth]{Figure2.png}\\
\caption{Methodology}
\label{fig:Figure 2}
\end{center}
\end{figure}



\pagebreak

\subsection{Scope of the Project}
Online scams affect a wide range of industries, services and the public in general. There are a large variety of online scams including money theft, identity theft crimes, accessing sensitive information, etc,. 

This project provides a wide range of applications and with slight modifications, finds scope in profusely many areas.\\
\textbf{Some of the areas carrying scope for this project are listed below.}

\subsubsection{Scope in Online Banking}
This project finds its principal application in securing online banking. In case of a theft of a user's login credentials, the thief can run off with the user's online funds. But with photo verification before transaction, the user will be notified of such an attempt of robbery and will have the power to seize control of his account by changing his credentials.
\subsubsection{Scope in protection against data theft}
Theft of sensitive data is another concern of many organisations and even public individuals. Organisations could have varied amounts of data about their products, employees, etc,. with high level of confidentiality. Such data banks are prone to online attacks for illegal access or tampering. Using photo validation in such data banks, the corresponding company officials would always be notified during any potential data breach and would have full control on providing access to the information.
\subsubsection{Scope in Identity theft cases}
Identity theft is a serious case of online scam where a fraud commits online crimes, while identifying as someone else. This project could be moulded to serve as a useful method to put a stop to identity theft in many areas of the internet. This system could be deployed on login pages of sites with high identity theft cases. When a scammer, imposing as someone else tries to login, the app takes a picture and sends it to the email of the account holder who can then verify whether his credentials are being put to misuse. After a scam is discovered, the user can change his credentials to secure his account. 

\pagebreak

\subsection{Project Organisation}
\vspace{1cm}
\subsubsection{Team Members}
\begin{enumerate}
    \itemsep0em
    \item \textbf{Harsh Bokan(12011009, CSE)}
    \item \textbf{Akshdeep Singh(12011026)}
    \item \textbf{Karanbir Singh Pannu(12011011)}
\end{enumerate}

\subsubsection{Technologies Used}
\begin{enumerate}
    \itemsep0em
    \item Google Meet(To work from remote locations)
    \item Canva
    \item GitHub
    \item \LaTeX
\end{enumerate}

\subsubsection{Workflow}
All the three team members worked remotely in this project. GitHub and Google meet proved very handy in cooperating with one another to the end of the project. This project called for a substantial command over both front-end and back-end web designing combined with certain techniques specific to this project. The exploration left us with extensive knowledge to determine the adaptability and flexibility of the project and even to dictate further research in the similar field. The workload was evenly distributed among the team members for efficacious direction of the project.

\subsubsection{Timeline}
The first week was dedicated to discovering an optimal method for the mitigation of online frauds. The following few weeks were devoted to understand the operation of back-end web development, mainly Nodejs, express, database management and photo capture. The major problems involving online frauds were also kept in prime consideration. Apart from these, the knowledge of VS Code and GitHub was key to implement the above tools and collaborate conveniently.
\pagebreak


\subsection{Summary}
\begin{itemize}
    \item \textbf{Purpose/Objective - }Online frauds and scams have become a problem of serious concern with the ever increasing frequency of online payments in almost every part of expenditure and service. These scams have caused the theft of enormous amount of currency and data from organisations and the general public. The motive of this project is to provide a safe and trustworthy gateway of online transactions where the end user gets notified every time his or her account is accessed, with photo identity and  carries the power to validate all the payments. This project will enable customers to safeguard their online funds against potential scams and thefts.
    \item \textbf{Design/Methodology/Approach - }The methodology of the project is plain sailing but fruitful in solving the concerned problem. The project includes a web application where a user logs into a bank account with the particular credentials. On entering the payment gateway, he or she enters the transaction details to execute the payment in someone's favor. Then the transactor is required to submit his/her picture through the web app which would be sent to the account owner via an email. The account owner, with the photo identity of the transactor, can then decide whether to allow or deny the ongoing transaction with a link in the email. If the owner validates the payment, the transaction is carried out successfully, otherwise it is cancelled. The owner is also provided with a link where he/she can reset their account password in case of an unauthorized attempt of access.
    \item \textbf{Scope of Project - }This project is designed for the foremost purpose of preventing scams in banking transactions using the online medium. In case of theft of a user's credentials, this project could help provide the owner with the identity of the thief. However this project could be shaped further to allow the end user to seize his/her account in case of a theft and also report the authorities of the same with a possible identity of the fraudster. This project can also be applied in data security to prevent unauthorised access or modification of data of an organization or individual. This project will incite further research in protection of users' data with the use of advanced technologies like ML and AI to provide facilities like real time document validation, etc.
    \item \textbf{Mitigation of Online Frauds - }With the help of this project, many people with deceitful intentions would be discouraged from tinkering with anyone's property, as they would be required to disclose their identity which could be used against them in a criminal proceeding. It also eliminates the threat of breaches by bots or viruses because they would have no facial identity to produce for the validation of the transaction. 
\end{itemize}
\pagebreak

%Chapter 2
\begin{Large}
\begin{center}
\textbf{CHAPTER\\
\section{STUDY AND REVIEW OF LITERATURE}}\\
\line(1,0){300}\\
\end{center}
\end{Large}
\vspace{0.5cm}
\subsection{Introduction}
The purpose of this chapter is to bring forth some outlines to solve the concerned problem of this project i.e., online frauds. The provided outlines include both, the practices which are currently being used in prevention of scams and those theorised but not yet put into effect.

Online scams impact a vast variety of organizations and services, due to the immense employment of online payments in said services. The different techniques mentioned target online frauds in specific areas and provide the corresponding solution. As the potency of an online scam could vary among different regions of target, the methods laid down could vary in their ability of decreasing the threat of an online fraud.

It should be noted the prime area of focus of this report is the problem of online stealing or banking frauds by misusing a customer's credentials, therefore the problems of identity theft, online harassment, ATM skimming for physical theft, etc,. are not as deeply explored.

The purpose of this chapter is to provide an outline of the different techniques that can be brought to use to diminish the threat of online fraud and not the detailed study of the said techniques. However, these techniques could be studied in detail following the references which provide the bedrock for this project.

This chapter is not limited to only one area of research in the field of online scams, because unlike before, a large number of systems have been made available to the common online servicing platforms, even those requiring high computing power, thanks to the evolution in technology. Many techniques followed in offline validation are now being taken advantage of in the online mode as well.

As this report mainly focuses on online banking frauds, the techniques in this chapter are given only an introduction since there is already a detailed literature on the provided methods.

This chapter is essentially made from a vast number of references from numerous sources which can be further explored to dig deeper in the listed tactics.
\pagebreak

\subsection{Prevailing methodologies used to prevent Online frauds}
\vspace{0.5cm}
\subsubsection{Questionnaire in survey}
Fraudulent behaviors have been predicted by the inconsistent answers provided by the participants\cite{PrevailingMethods1}. The survey conducted by a software checks for proper/consistent answers while it contains some strange or personal questions about social desirability and at the same time collects paradata of the subject(timestamp on questions, movement of mouse, etc,.). This method can detect bots, assess personality traits associated with possible fraudsters, however there is some ethical issues on whether this information should be released or not. The survey would be hard to abandon or resubmit once started, and order of questions would be shuffled with each administration. The survey can further be used to collect IP addresses of devices used to detect encryptions, frequency of survey by one device and also ban IPs of confirmed fraudsters from participating. The cons being that subjects could skip the survey due to discomfort and it only makes frauds difficult and does not prevent them.

\subsubsection{Hidden Markov Model}
The Hidden Markov Model(HMM) is a statistical model in which the system being modeled is assumed to be a Markov model\cite{PrevailingMethods2}. The ranges of transaction amount and types of items are fed to the HMM as observation symbols and states respectively. It is based on the baum-welch algorithm and is initially trained with the normal behavior of account holders. If an incoming transaction is declared fraudulent by the HMM model with high probability, a One Time Password(OTP) is sent to the mobile number for verification. The accuracy of this system was calculated to be around 72\% over a wide variation of input data.


\subsubsection{Fraud Detection Based on Local and Global Behavior}
This research\cite{PrevailingMethods3} provides a system for online fraud detection in real time using two complementary techniques. First is the differential analysis approach which compares account usage to normal history of the user and any deviation would indicate a potential fraud. Second is the global analysis approach where each device is monitored and classified either as legitimate or fraudulent. The model keeps an eye for suspected behavior as large number of accounts accessed by a single fraudster, small transactions made in many accounts, increased number of password failures before a fraud. It maintains a suspect list which contains devices with fraudulent behaviors. These list items are then explicitly deemed as genuine or fraudulent. The effective identification is made by a component installed on every device during its first access to the bank. A monitor counts the number of different accounts accessed by each device and the data collected is used by the global analysis module to predict the possibility of a transaction being a fraud.

\subsubsection{Using financial intelligence to target fraud victimisation}
This study\cite{PrevailingMethods4} based in Australia explains the emergence of a victim oriented approach to target online frauds, where police and consumer protection agencies put financial intelligence at work to identify potential online fraud victims. Operation Sunbird(2012) targeted advance fee frauds and romance fraud victimisation where a victim sends a small amount of money hoping to receive a larger sum in return. It is a five stage process. First stage is intelligence where WAPOL(West Australian Police) use intelligence to find potential victims by screening potential transactions. Second stage is intervention where Commerce sends a letter to the household warning the victim of a high possibility of fraud. Third stage is intervention where Commerce liaises with bank to block transactions of both the potential fraud and victim. Fourth stage involves intelligence where Commerce gathers and analysis intelligence gathered by investigation and letter recipient who made contact. The final stage is investigation whereby WAPOL apprehends possible suspects and refers the intelligence of corresponding countries about the fraud.

\subsubsection{Hierarchical attention mechanism}
This study\cite{PrevailingMethods5} puts to use, an attention based classifier for real time risk assessment of each individual transaction in the form of a fraud probability. This mechanism consists of two main components. First component entails embedding of categorical features in a continuous space and uniting those characteristics into one vector using an attention mechanism. The second component is responsible to expose fraudulent activity carried out by a sequence level attention. Tested over 26.1 million transactions over six months, this model outperformed LSTM(Long Short-Term Memory) proving  that there is substantial benefit in attention based models that dynamically assign weights to transactions.

\subsubsection{Embedded Training Email System}
Phishing attacks involve criminals disguise as legit websites to lure users into providing their personal information or releasing some malware into their system. This study\cite{PrevailingMethods6} deals with training people about phishing attacks. Users are communicated to a website through periodic emails which could be a potential phishing attempt, and then they are tested whether they would disclose their information or not. The medium of email is chosen as most of phishing attacks occur through email. The study also explains the development of a game-based approach to train people to identify phishing sites.

\pagebreak

\subsection{Summary}
This chapter brings forward the available and theorised technologies used to detect and prevent online frauds. The methods mentioned have been taken different researches carried out for online fraud prevention.

Online theft and crimes is a very vast field and is understudied in many respects. It impacts the society at many points and demands serious attention. These studies are based on the different areas of impact of online fraudsters and therefore a different degree of efficiency in their execution. 

The purpose of this chapter is to lay out a thorough but not all-inclusive review of the literature of the different techniques discussed. However these studies could be investigated further through the reference provided at the end. 

The methodologies presented in this chapter communicate many ways for the prevention of online scams ex, survey through questions to detect fraudulent behavior, analysing behavioral patterns of devices and users accessing customer's account to find out any suspicious activity through HMM, local and global behavior, hierarchical attention, etc,. Methods of educating the users against such potential crimes and monitoring and helping victims were also discussed as mitigation techniques. 


\pagebreak


%Chapter 3
\begin{Large}
\begin{center}
\textbf{CHAPTER\\
\section{IMPLEMENTATION OF PHOTO VALIDATION}}\\
\line(1,0){300}\\
\end{center}
\end{Large}
\vspace{0.5cm}
\subsection{Introduction}
This report aims at the designing of a secure payment gateway where a user has complete control over the transactions from his account. This goal is achieved through identity validation of the transactor in the form of photo verification. In this way the user can detect a suspicious transaction and therefore protect herself or himself from a fraud or online theft.

This chapter concentrates on the implementation of photo verification to create a safe payment gateway and protect the user from online frauds. All the techniques and methods used to bring this project to life are enlisted in this chapter from the front end through which the user interacts with the web app and the back end where all the processes of form validations, photo capturing and verification occur.

This chapter contains various flowcharts and diagrams to get an overview and explain the methodology and overall working of the project in a simplified way

Further, the screenshots of the web application in working are attached to get a passive perspective of the working of the web app along with the written explanation of every picture right below.

Online scams are a problem of grave concern and puts the hard earned income of an individual at the risk of theft. This project would prove to be a safe gateway which would thwart any such attempt of theft or scam and keep the owner notified of every such access on his account.
\pagebreak

\subsection{Case Diagram}
\vspace{2cm}
\begin{figure}[H]
\begin{center}
\hspace*{-0.9in}
\includegraphics[width=1.3\textwidth]{Figure3.png}\\
\caption{Use Case Diagram}
\label{fig:Figure 3}
\end{center}
\end{figure}

\begin{center}
The above figure shows the use of Case diagram for the web application.
\end{center}

\pagebreak

\subsection{List of main Classes under Use Cases}
\vspace{0.5cm}
\begin{enumerate}
    \itemsep0em
    \item Get\_input()
    \item Authenticate()
    \item Send\_mail()
    \item mongo\_data()
    \item cookies()
\end{enumerate}

\subsection{Class Diagram}


\begin{figure}[H]
\begin{center}
\includegraphics[width=1\textwidth]{Classes.png}\\
\caption{Use Class Diagram}
\label{fig:Figure 4}
\end{center}
\end{figure}

\begin{center}
The above diagram shows the Class Diagram from Use Cases.
\end{center}

\pagebreak

\subsection{Method of Implementation}
\vspace{0.5cm}
\begin{figure}[H]
\begin{center}
\includegraphics[width=0.7\textwidth]{Implementation.png}\\
\caption{Flowchart of Application}
\label{fig:Figure 5}
\end{center}
\end{figure}

The above figure shows the brief working of the application in pictorial form.


\pagebreak

\subsection{Front End Development}
\vspace{0.5cm}
\subsubsection{Login Page}
This is the first page the user interacts with. It is elegant in design yet simple to use. It contains input boxes for username and password, a link to the sign up page and a submit button.

\subsubsection{Sign-up Page}
The sign up page is similar to the login page. It has three input boxes for the username, password and email and a submit button.

\subsubsection{Payment Details Page}
Here the user enters the details of the benefactor of the transaction. It contains input boxes for payee name, account number, IFSC code and the amount to be transferred. At the end there is a submit button which takes the user to the webcam page.

\subsubsection{Webcam Page}
As the page opens, the site asks the user to allow access to his/her webcam. The user can see his live image in this page on the left box. There are two buttons, namely Capture and Submit. Capture captures the image and displays it in the right box. Then submit sends the user to the waiting page.

\subsubsection{Waiting Page}
As the name suggests, this page is built for the user to wait for another process to complete. It has a graceful animated gif of an email and says 'Email has been sent on top'.

\subsubsection{Verify Page}
This page is served to the end user through an email link. It contains the details of the payment, picture of the transactor, a form with three bullet points and button saying verify. The options given are to verify or decline the payment, or reset the password.

\subsubsection{Password Reset Page}
This page is for the end user too, to change his/her  password. It is a form with three input boxes for the old password, new password and to re enter the new password. It has a submit button at the end.

\subsubsection{Confirmed Page}
This page is meant for both the transactor and the account owner indicating the correct completion of task.

\subsubsection{Declined Page}
This page is served to the transactor and the end user when their request is declined.

\subsubsection{Error Page}
This page is only for the transactor when request times out. It has a nice animation and shows the 404 error code.

\subsection{Backend Development}
\vspace{0.5cm}
The whole methodology of the project is carried out by the backend. When the user logs in, the database is checked for those credentials and the user is allowed access to the pages further in. A cookie is stored in the browser to identify the user throughout the payment. At the webcam page, the picture of the transactor along with the details of the payment are stored in the database as a transaction request. This information is then extracted to display it to the end user via email. The end user validates the payment which updates the payment status on the database. This database is then checked on the transactor's side to show him/her the status of the payment. Then the tokens for login and email session are erased so they cannot be accessed again without placing another request.

To ensure the safety of the transaction and secure mapping of a transaction to an email, the following token authentication methods have been used:

\subsubsection{Login Token}
When the user enters the correct credentials, a random hundred character long token is assigned to the transaction and stored in the cookies and the database for their communication. When any page is accessed, the backend checks if there is a record present in the database corresponding to the login token in the cookies. If yes, access in granted, otherwise the user has to log in again. At the end of the transaction, the login token in database is changed to 'logged-out' and is further deleted from the cookies. So a login token is usable only once and its value changes every time making a secure login system.

\subsubsection{Email Session Token}
Another concern is to bind an email with a transaction so that it can only affect that particular transaction and is rendered useless after one use. This is achieved by another token for the email. When the mail is sent, a token is sent in the verification link of the email as a query. When the link on the email is clicked, the backend checks for a record in the database for the token. Only if it finds an active record in the database(where payment is still pending), does it give access to the email, otherwise it shows that this session has expired. After the email has been verified, the session token is changed to 'logged-out' and that email cannot access that record thereafter.\\

Apart from this, the full details of every transaction are stored in the database which can be accessed by the bank in case of a fraud. The end user also has the picture of every transactor in her/his emails.

\pagebreak

\begin{center}

\vspace{1cm}
The below figure shows the backend processes of the web app.
\end{center}


\begin{figure}[H]
\begin{center}
%\hspace*{-0.1in}
\includegraphics[width=1\textwidth]{Figure4.png}\\
\caption{Backend Flowchart}
\label{fig:Figure 6}
\end{center}
\end{figure}

\pagebreak


\subsection{Test and Results}
\vspace{0.5cm}
\begin{figure}[H]
\begin{center}
%\hspace*{-0.1in}
\includegraphics[width=1\textwidth]{Login.png}\\
\caption{Login Page}
\label{fig:Figure 7}
\end{center}
\end{figure}

This is the first page of the website. Here the user enters her/his login credentials to log in.
\vspace{0.7cm}

\begin{figure}[H]
\begin{center}
%\hspace*{-0.1in}
\includegraphics[width=1\textwidth]{Payment.png}\\
\caption{Payment Details Page}
\label{fig:Figure 8}
\end{center}
\end{figure}

After logging in, the transaction details of the benefactor are entered.

\pagebreak

\begin{figure}[H]
\begin{center}
%\hspace*{-0.1in}
\includegraphics[width=1\textwidth]{Webcam.png}\\
\caption{Webcam Page}
\label{fig:Figure 9}
\end{center}
\end{figure}

The transactor clicks his/her picture by clicking the button on the left. It is then submitted for verification by the submit button on the right.

\vspace{0.7cm}

\begin{figure}[H]
\begin{center}
%\hspace*{-0.1in}
\includegraphics[width=0.7\textwidth]{Email.png}\\
\caption{Received email}
\label{fig:Figure 10}
\end{center}
\end{figure}

This is the email received by the end user. It contains the details of the payment along with the picture of the transactor.

\pagebreak

\begin{figure}[H]
\begin{center}
%\hspace*{-0.1in}
\includegraphics[width=0.7\textwidth]{Verify.png}\\
\caption{Verification Page}
\label{fig:Figure 11}
\end{center}
\end{figure}

The verify link in the email leads to this page. Here the account owner can select any of the three options for the ongoing transaction.

\vspace{0.7cm}

\begin{figure}[H]
\begin{center}
%\hspace*{-0.1in}
\includegraphics[width=1\textwidth]{Reset.png}\\
\caption{Password Reset Page}
\label{fig:Figure 12}
\end{center}
\end{figure}

If the end user chooses to reset her/his password, she/he is directed to this page. User enters the old password and the new password to change the login password.

\pagebreak

\begin{figure}[H]
\begin{center}
%\hspace*{-0.1in}
\includegraphics[width=1\textwidth]{Declined.png}\\
\caption{Payment Declined Page}
\label{fig:Figure 13}
\end{center}
\end{figure}

As the end user chose reset password, the ongoing payment is declined and the transactor sees this page. Had the end user chosen 'Decline payment', the transactor would still have received the same page.

\vspace{0.7cm}

\begin{figure}[H]
\begin{center}
%\hspace*{-0.1in}
\includegraphics[width=1\textwidth]{Confirmed.png}\\
\caption{Request Confirmed Page}
\label{fig:Figure 14}
\end{center}
\end{figure}

After the end user has completed the verification process, he is taken to the confirmation page given above.

\pagebreak

\textbf{Exploring other options:}

\begin{figure}[H]
\begin{center}
%\hspace*{-0.1in}
\includegraphics[width=1\textwidth]{Successful.png}\\
\caption{Payment Successful Page}
\label{fig:Figure 15}
\end{center}
\end{figure}

If the end user confirms the payment, the transactor is taken to this page.

\vspace{0.7cm}

\begin{figure}[H]
\begin{center}
%\hspace*{-0.1in}
\includegraphics[width=1\textwidth]{Expired.png}\\
\caption{Session Expired Page}
\label{fig:Figure 16}
\end{center}
\end{figure}

If the end user opens the link in an email which has already verified a payment, the above page opens showing that the session has already expired.

\pagebreak

\begin{figure}[H]
\begin{center}
%\hspace*{-0.1in}
\includegraphics[width=1\textwidth]{Error.png}\\
\caption{Payment Timed Out Page}
\label{fig:Figure 17}
\end{center}
\end{figure} 

If the end user does not reply to the email within three minutes of sending the mail, the payment is canceled automatically and the transactor sees this page.

\vspace{0.7cm}

\begin{figure}[H]
\begin{center}
%\hspace*{-0.1in}
\includegraphics[width=1\textwidth]{Atlas.png}\\
\caption{Transaction stored in database}
\label{fig:Figure 18}
\end{center}
\end{figure} 

As already mentioned, every transaction is stored in mongo atlas database. The above figure shows the document for the transaction carried out above.



\pagebreak

%Chapter 4
\begin{Large}
\begin{center}
\textbf{CHAPTER\\
\section{CONCLUSION AND FUTURE DIRECTIONS}}\\
\line(1,0){300}\\
\end{center}
\end{Large}
\vspace{0.5cm}
\subsection{Conclusion}
Online scams have penetrated almost every aspect of businesses and services and its origin itself lies with the advent of digitalisation of these amenities. Millions all around the world are victimized by fraudsters every single day bringing about loss of prodigious amount of finances. The ever increasing popularity of online gateways for transactions, payments, shopping, etc,. is one of the prime pull factors for fraudsters to carry out such felonies. To detect such frauds and scams is yet another strain due to fraudsters adapting to the newer countermeasures designed to throw them out. Although this report does not focus on putting a stop on online crimes altogether, it does present an ingenious strategy to safeguard the general public from such frauds in one mode. Photo verification is an uncomplicated, facile yet reliable and adaptable solution to the problem discussed above. It is the right blend of technology and the key element of human intervention assuring its solidity.

This chapter of the report discusses the overall purpose of the system of securing transactions, limitations of the project and its scope in the future. Online scams are a serious problem but are unexpectedly understudied. These frauds disturb the economic working of the world and disrupts the interaction of the systems within. These scams have broad social impacts like loss of trust in people, commerce or government institutions. It may instil fear and loss of confidence in the commercial markets. They also impact the mental well-being of the victim and exert negative impacts on self-esteem and relationship with others.

The method discussed in this report relies on human intelligence and therefore would prove trustworthy unlike models using machine learning which are dependent on the nature of the limited amount of dataset fed to them. This ingredient of human involvement is what distinguishes this model from the ones already present. It could further be improved for a broader range of application if given the right direction.

\pagebreak

\subsection{Limitations and Future Directions}
\vspace{0.5cm}
\subsubsection{Limitations}
Although the model given in this report proves to be effective, it is limited regarding the relief it provides to the underlined problem. The prime limitation is the range of devices which can run this system of verification. As this involves taking the picture of the transactor at the time of transaction, only devices with a camera like phone, laptops and desktops with webcams could run this model. Many desktop users may not have a webcam with their computer and therefore would be unable to verify themselves during payments. Apart from the availability, factors like the quality of camera and internet connection also affect the performance.

Another limitation of this project is the place and time of the person making a payment. A clearly recognizable picture would require good lighting conditions, preferably during the day or external lighting at night. Now if the transactor is not at a suitable place, it could affect the effectiveness of the model.

Even though this project is designed to keep fraudsters at bay, they are brisk at adapting to the countermeasures planted against them. In this case, tampering with the picture may bring down the efficacy of the system. Acts like covering the camera or holding a picture of someone else in front of it could make it impossible for the account holder to identify the transactor or even effectively trick him/her into validating the payment resulting in a fraud.

\subsubsection{Future Directions}
 This project aims to solve the problem of scams involving online transactions by the conscious presence of human judgement. As the pressure on the system increases, the project could be moulded with the help of technology to display faster handling while on the same time, not compromising with the reliability of the model.
 
 To handle a large amount of traffic due to multiple people accessing data servers of some organizations, the system can be combined with an AI model for facial recognition, to itself verify the transactor and thereby allow or decline access. The facial recognition could be used with photo verification by a human to make sure that the person submitting the picture is actually present in front of the camera and is not holding a picture of someone the account holder may trust. This could be achieved using an AI which tracks specific points of the face to detect eye blinks, stretching of the skin, etc., to confirm that the feed being given to the webcam is live. This system could be trained to sharpen the image according to the surrounding for better recognition. With the consent of the user, the AI could be trained the behavioral patterns conveyed by the transactor to learn the general behaviour of a genuine user and a fraud.
 
 Although this project was made to create a safe payments gateway, its implementation runs beyond that. As already explained in previous chapters, this model could be used to safeguard data from unauthorised access by situating this system at login or sign-in pages of a company.
 
 Another direction this project can take is safeguarding users' accounts on different platforms on the internet. If this system of verification is applied on login pages of social media sites, the genuine user would know if someone is trying to access the internet with their identity. It would therefore provide immunity against identity theft cases.


\pagebreak

\begin{thebibliography}{}
\bibitem{Project_Objective}
Shewangu Dzomira, Ph.D., Department of Finance, Risk Managementand Banking, College of Economics & Management Sciences, Univer-sity of South Africa, South Africa.
\bibitem{PrevailingMethods1}
Teitcher, J. E., Bockting, W. O., Bauermeister, J. A., Hoefer, C. J., Miner, M. H., & Klitzman, R. L. (2015). Detecting, preventing, and responding to "fraudsters" in internet research: ethics and tradeoffs. The Journal of law, medicine & ethics : a journal of the American Society of Law, Medicine & Ethics, 43(1), 116–133. https://doi.org/10.1111/jlme.12200
\bibitem{PrevailingMethods2}
Mhamane S, Lobo L.M.R.J. Use of Hidden Markov Model as Internet Banking Fraud Detection, International Journal of Computer Applications (0975 – 8887), Volume 45– No.21, May 2012
\bibitem{PrevailingMethods3}
Kovach, Stephan & Ruggiero, W.V.. (2011). Online Banking Fraud Detection Based on Local and Global Behavior. In: Proc. of the Fifth International Conference on Digital Society.
\bibitem{PrevailingMethods4}
Cassandra Cross (2016) Using financial intelligence to target online fraud victimisation: applying a tertiary prevention perspective, Criminal Justice Studies, 29:2, 125-142, DOI: 10.1080/1478601X.2016.1170278
\bibitem{PrevailingMethods5}
Achituve, Idan & Kraus, Sarit & Goldberger, Jacob. (2019). Interpretable Online Banking Fraud Detection Based On Hierarchical Attention Mechanism. 1-6. 10.1109/MLSP.2019.8918896. 
\bibitem{PrevailingMethods6}
Cranor P.K.L, Hong Y.W.R.J, Nunge A.A.E.(2006) Protecting People from Phishing: The Design and Evaluation of an Embedded Training Email System, CyLab Carnegie Mellon University Pittsburgh(November 9, 2006)
\end{thebibliography}

\end{document}